% test dokumentu
\documentclass{beamer}
\usepackage{tikz}
\usepackage{hyperref}
\usepackage{multimedia}
\usepackage[polish]{babel}% Język
\let\babellll\lll
\let\lll\relax% Naprawia błąd \lll
\usepackage{polski}% Język
\usepackage[utf8]{inputenc}% Kodowanie
\usetheme{PaloAlto}
\usecolortheme{crane}
\title{Historia internetu}
\subtitle{od A do Z}
\author[Mateusz Szewczak]{Mateusz Szewczak
	\texttt{Politechnika Koszalińska, kierunek Informatyka, grupa I10}}
\date[PL’20]{06.11.2020}
\begin{document}
	\bibliographystyle{unsrt}
	\begin{frame}
		\maketitle
	\end{frame}
	\begin{frame}{Spis treści}
		\tableofcontents
	\end{frame}
	\begin{frame}{Jak to się zaczęło?}
		\section{Początki internetu}
		\subsection{Komunikacja między komputerami}
		Na początku lat 60 ubiegłego wieku, po zaprezentowaniu kilku pierwszych komputerów, powstał pomysł, aby stworzyć połączenia między komputerami. Idee opisał i zrealizował Paul Baran z firmy Rand Corporation
		\nocite{hist:int:wiki}
		
	\end{frame}
	\begin{frame}{Bibliografia}
	\section{Bibliografia}
	\bibliography{bibliografia.bib}
	\end{frame}
\end{document}